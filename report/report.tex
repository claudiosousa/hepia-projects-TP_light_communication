\documentclass[11pt, a4paper]{article}

% Packages
\usepackage[francais]{babel}
\usepackage[T1]{fontenc}
\usepackage[utf8]{inputenc}

\usepackage[left=2cm, right=2cm, top=2cm, bottom=2cm]{geometry}
\usepackage{fancyhdr}
\usepackage{lastpage}
\usepackage{hyperref}
\usepackage{float}
\usepackage{graphicx}
\graphicspath{{./img/}}

% Reset paragraph indentation -------------------------------------------------
\setlength{\parindent}{0cm}

% Page header and footer ------------------------------------------------------
\pagestyle{fancy}
\setlength{\headheight}{14pt}
\renewcommand{\headrulewidth}{0.5pt}
\lhead{Programmation temps-réel}
%\lhead{\includegraphics[height=1cm]{logo.jpg}} % Change \headheight to right size
\chead{Émetteur et récepteur lumineux}
\rhead{Claudio Sousa, David Gonzalez}
\renewcommand{\footrulewidth}{0.5pt}
\lfoot{17/05/2018}
\cfoot{Groupe 3}
\rfoot{Page \thepage /\pageref{LastPage}}

% Table of contents depth -----------------------------------------------------
\setcounter{tocdepth}{3}

% Document --------------------------------------------------------------------
\begin{document}

\title
{
    \Huge{Programmation temps-réel} \\
    \Huge{Émetteur et récepteur lumineux}
}
\author
{
    \LARGE{Claudio Sousa, David Gonzalez - Groupe 3}
}
\date{17/05/2018}
\maketitle

\thispagestyle{empty}

%\tableofcontents

\newpage

% -----------------------------------------------------------------------------
\section{État du projet}

Le projet est en état de marche et
toutes les fonctionnalités ont pu être implémentées. \\

L'émetteur fonctionne sans soucis.

Les 4 tâches du récepteur fonctionnent également bien, y compris sous forte charge.

\section{Anomalies ou bugs}

Les programmes fonctionnent sans anomalie.

\section{Traces sous forte charge}



\end{document}
